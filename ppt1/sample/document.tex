\documentclass{beamer}
\usepackage{bookmark}
\usepackage{graphics}
\usepackage{amsmath}
\usepackage{amssymb}
\usepackage{amsfonts}
\usepackage{multirow}
\usepackage{amsthm}
\usepackage{setspace}
\usepackage{etoolbox}
\usepackage{diagbox}
\usepackage{listings}
% This is the file main.tex
\usetheme{Warsaw}
\setbeamertemplate{theorems}[numbered]
\title{Title}
\subtitle{subtitle}
\author{Author}
\institute{
	Univ. of Sci. \& Tech. of China
}
\date{\today}
\begin{document}
	\begin{frame}
		\titlepage
	\end{frame}
	\begin{frame}
		\tableofcontents
	\end{frame}
\section{Introduction}
\subsection{Background}
\begin{frame}{Individual Causal Effect}
	Individual causal effects are defined as a contrast of the values of
	counterfactual outcomes. 
\begin{block}{Counterfactual outcome}
	Such as $Y^{a=1}$ and $Y^{a=0}$, which may not actually ((occur. 
	
	Only one of those outcomes is observed for each individual, and all
	other counterfactual outcomes remain unobserved.
\end{block}
	This paper only considers Neyman's model where each subject has only two potential responses.
	\end{frame}
\begin{frame}{Average Causal Effect}
	
	\begin{block}{The intention-to-treat parameter ($b_{ITT}$)}
			$b_{ITT}$ is the average response if all subjects are assigned to treatment, minus the average response if all subjects are assigned to control.
		\end{block}
	 
	  Sometimes $b_{ITT}$ is called the average causal effect or the average treatment effect. 
	  \begin{block}{Remark}
	  	$b_{ITT}$ needs to be estimated!
	  \end{block}
\end{frame}
\begin{frame}{Untenable Model Asumptions}
	\begin{itemize}
		\item In randomized or obeserved experiments for causal inference, data are often analyzed using regression models. 
		\begin{equation*}
		\end{equation*}
		\item Regression model needs its assumptions but randomization does not justify regression models. 
		
		In fact the assignment variable (to treatment or control) and the error term in the model will generally be strongly related.
		\begin{equation*}
		\end{equation*}
		\item Wrong model assumptions usually bring wrong conlusions!
	\end{itemize}
\end{frame}
\end{document}